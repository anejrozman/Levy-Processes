\documentclass[]{beamer} %zaradi [handout] \pause ne dela
\usepackage[T1]{fontenc}
\usepackage[utf8]{inputenc}
\usepackage[slovene]{babel}
\usepackage{pgfpages}
\usepackage{amsmath}
\usepackage{amssymb}
\usepackage{colortbl}
\usepackage{tikz}
\usepackage{array}
\usepackage{amsmath,amsthm, amsfonts,amssymb}
\usepackage{mathtools}

\setbeameroption{hide notes}
%\setbeameroption{show notes on second screen=right}

\mode<presentation>
\usetheme{Berlin}
\useinnertheme[shadows]{rounded}
\useoutertheme{infolines}
\usecolortheme{dolphin}
\usepackage{palatino}
\usefonttheme{serif}

%okolja za izreke, definicije, ...
\theoremstyle{plain}
\newtheorem{izrek}{Izrek}
\newtheorem{definicija}{Definicija}
\newtheorem{trditev}{Trditev}
\newtheorem{posledica}{Posledica}
\newtheorem{opomba}{Opomba}
\newtheorem{zgled}{Zgled}

%\beamertemplatenavigationsymbolsempty
\setbeamertemplate{headline}{}
%\setbeamertemplate{footline}{}

\title{Lévijevi procesi in njihova uporaba v financah}
\subtitle{}
\author[Anej Rozman]{Anej Rozman}
\institute[]{Mentor: doc.~dr. Martin Raič}
\date[]{}

\begin{document}


\frame{\titlepage}


\begin{frame}
  \frametitle{Neskončno deljive porazdelitve}
  \begin{definicija}
    Pravimo, da ima realno številska slučajna spremenljivka $X$ \textit{neskončno deljivo porazdelitev}, če
    za vsak $n \in \mathbb{N}$ obstaja zaporedje neodvisnih enako porazdeljenih slučajnih spremenljivk $(X_i)_{i=1,\dots,n}$, 
    da velja 
    $$
    X \stackrel{d}{=} X_1 + X_2 + \dots X_n,
    $$
    kjer $\stackrel{d}{=}$ pomeni enakost v porazdelitvi.
\end{definicija}


  \end{frame}


\begin{frame}
  \frametitle{Nekaj zgledov}
  \begin{zgled}
    Normalna porazdelitev je neskončno deljiva.
  \end{zgled}
  \begin{center}
    Naj bo $X \sim N(\mu, \sigma^2)$. Vemo da so linearne kombinacije neodvisnih normalno 
    porazdeljenih slučajnih spremenljivk spet normalno porazdeljene. Torej lahko za poljuben 
    $n\in\mathbb{N}$ zapišemo
    $$
      X \stackrel{d}{=} N(\frac{\mu}{n}, \frac{\sigma^2}{n}) + \dots + N(\frac{\mu}{n}, \frac{\sigma^2}{n}).
    $$
  \end{center}
\end{frame}


\begin{frame}
  \frametitle{Nekaj zgledov}
  \begin{zgled}
    Poissonova porazdelitev je neskončno deljiva.
  \end{zgled}
  \begin{center}
    Naj bo $X \sim \text{Pois}(\lambda)$. Vemo da so linearne kombinacije neodvisnih Poissonovo
    porazdeljenih slučajnih spremenljivk spet normalno porazdeljene. Torej lahko za poljuben 
    $n\in\mathbb{N}$ zapišemo
    $$
      X \stackrel{d}{=} \text{Pois}(\frac{\lambda}{n}) + \dots + \text{Pois}(\frac{\lambda}{n}).
    $$
  \end{center}
\end{frame}

\begin{frame}
  \frametitle{Compound Poisson process}
\end{frame}


\begin{frame}
  \begin{izrek}
    \textbf{(Lévy-Hinčinova formula)} Verjetnostna mera $\mu$ na realni osi je neskončno deljiva s 
    karakterističnim eksponentom $\Phi$,
    $$
    \int_{\mathbb{R}} e^{i\theta x} \mu(dx) = e^{-\Phi(\theta)},\ \text{za} \ \theta \in \mathbb{R},
    $$
    če in samo če obstaja taka trojica $(a, \sigma, \Pi)$, kjer sta $a, \sigma \in \mathbb{R}$ in $\Pi$
    mera na $\mathbb{R}\backslash\{0\}$, ki zadošča $\int_{\mathbb{R}}1 \wedge x^2 \Pi(dx)\leq \infty$, 
    da za vsak $\theta \in \mathbb{R}$ velja
    $$
    \Phi(\theta) = ia\theta + \frac{1}{2}\sigma^2\theta^2 + \int_{\mathbb{R}}(1 - e^{i\theta x} + i\theta x {1}_{(|x|<1)})\Pi(dx).
    $$
    Še več, trojica $(a, \sigma, \Pi)$ je enolično določena.
  \end{izrek}
\end{frame}

\end{document}