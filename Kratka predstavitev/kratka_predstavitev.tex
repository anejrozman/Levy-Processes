\documentclass[]{beamer} %zaradi [handout] \pause ne dela
\usepackage[T1]{fontenc}
\usepackage[utf8]{inputenc}
\usepackage[slovene]{babel}
\usepackage{pgfpages}
\usepackage{amsmath}
\usepackage{amssymb}
\usepackage{colortbl}
\usepackage{tikz}
\usepackage{array}
\usepackage{amsmath,amsthm, amsfonts,amssymb}
\usepackage{mathtools}
\usepackage{dsfont}

\setbeameroption{hide notes}
%\setbeameroption{show notes on second screen=right}

\mode<presentation>
\usetheme{Berlin}
\useinnertheme[shadows]{rounded}
\useoutertheme{infolines}
\usecolortheme{seahorse}
\usepackage{palatino}
\usefonttheme{serif}

%okolja za izreke, definicije, ...
\theoremstyle{plain}
\newtheorem{izrek}{Izrek}
\newtheorem{definicija}{Definicija}
\newtheorem{trditev}{Trditev}
\newtheorem{posledica}{Posledica}
\newtheorem{opomba}{Opomba}
\newtheorem{zgled}{Zgled}
\newtheorem{lema}{Lema}

%\beamertemplatenavigationsymbolsempty
\setbeamertemplate{headline}{}
%\setbeamertemplate{footline}{}

\title{Lévijevi procesi in njihova uporaba v financah}
\subtitle{}
\author[Anej Rozman]{Anej Rozman}
\institute[]{Mentor: doc.~dr. Martin Raič}
\date[]{}

\begin{document}


\frame{\titlepage}

\begin{frame}
  \frametitle{Lévijev proces}
  \begin{definicija}
      Slučajnemu procesu $X = \{X_t \mid t \geq 0\}$ definiranem na verjetnostnemu
      prostoru $(\Omega, \mathcal{F}, \mathds{P})$ pravimo \textit{Lévijev proces}, če zadošča naslednjim pogojem:
      \begin{enumerate}
          \item $\mathds{P}(X_0 = 0)=1$.
          \item Trajektorije $X$ so $\mathds{P}$-skoraj gotovo zvezne z desne (z levimi limitami).
          \item Za $0 \leq s \leq t$ je $X_t - X_s$ enako porazdeljena kot $X_{t-s}$.
          \item Za $0 \leq s \leq t$ je $X_t - X_s$ neodvisna od $\{X_u \mid 0 \leq u \leq s\}$.
      \end{enumerate}
  \end{definicija}
\end{frame}

\begin{frame}
  \frametitle{Neskončno deljive porazdelitve}
  \begin{definicija}
    Pravimo, da ima realno številska slučajna spremenljivka $X$ \textit{neskončno deljivo porazdelitev}, če
    za vsak $n \in \mathbb{N}$ obstaja zaporedje neodvisnih enako porazdeljenih 
    slučajnih spremenljivk $\left(X_i\right)_{i=1,\dots,n}$, da velja
    $$
    X \stackrel{d}{=} X_1 + X_2 + \dots X_n,
    $$
    kjer $\stackrel{d}{=}$ pomeni enakost v porazdelitvi.
\end{definicija}


  \end{frame}

\begin{frame}
  \frametitle{Nekaj zgledov}
  \begin{zgled}
    Poissonova porazdelitev je neskončno deljiva.
  \end{zgled}
  \pause
    Naj bo $X \sim \text{Pois}(\lambda)$. Vemo da so vsote neodvisnih Poissonovo
    porazdeljenih slučajnih spremenljivk spet Poissonovo porazdeljene. Torej lahko za poljuben 
    $n\in\mathbb{N}$ zapišemo
    $$
      X \stackrel{d}{=} \text{Pois}(\frac{\lambda}{n}) + \dots + \text{Pois}(\frac{\lambda}{n}).
    $$
  \pause
  Z enakim razmislekom lahko pokažemo, da je Normalna porazdelitev neskončno deljiva.
\end{frame}

\begin{frame}
  \frametitle{Nekaj zgledov}
  \begin{zgled}
    $ X\sim \textit{U}([0, 1])$ ni neskončno deljiva.
  \end{zgled}
  \pause
  Recimo, da je $X$ neskončno deljiva. Potem za dani $n \in \mathbb{N}$ obstajajo
  neodvisne enako porazdeljene slučajne spremenljivke $X_1, \dots, X_n$, da velja
  $$
  X \stackrel{d}{=} X_1 + \dots + X_n.
  $$
  \pause
  <--- Naprej na tabli



\end{frame}


\begin{frame}
  \begin{lema}
    Linearne kombinacije neodvisnih neskončno deljivih slučajnih spremenljivk so neskončno deljive slučajne spremenljivke.
\end{lema}

\pause
\begin{proof}
    Naj bodo $X_1, X_2, ..., X_m$ neodvisne neskončno deljive s. s. Tedaj lahko za vsak $n \in \mathbb{N}$ 
    zapišemo $X_i \stackrel{d}{=} X_{i_1} + \dots + X_{i_n}$, torej za
    $a_1, \dots, a_m \in \mathbb{R}$ lahko $a_1X_1 + \dots + a_mX_m$ zapišemo kot 
    \begin{align*}
        a_1X_1 + \dots + a_mX_m &\stackrel{d}{=} \\
                                &\stackrel{d}{=} a_1(X_{1_1} + \dots + X_{1_n}) + \dots + a_m(X_{m_1} + \dots + X_{m_n}) \\
                                &\stackrel{d}{=} \sum_{i=1}^ma_iX_{i_1} + \dots + \sum_{i=1}^ma_iX_{i_n}.
    \end{align*}
\end{proof}
\end{frame}


\begin{frame}
  \frametitle{Lévy-Hinčinova formula}
  \begin{izrek}
    \textbf{(Lévy-Hinčinova formula)} 
    Neskončno deljive porazdelitve na $\mathbb{R}^+$ je možno opisati s pari $(\sigma, \nu)$, kjer je 
    $\sigma \in \mathbb{R}^+$ in $\nu$ mera, ki zadošča pogoju $\int_{\mathbb{R}\backslash\{0\}}1 \wedge x^2 \nu(dx)< \infty,$
    in sicer paru $(\sigma, \nu)$ priredimo karakteristično funkcijo $\varphi_{\sigma, \nu}(t) = e^{-\theta(t)}$,
    $$
    \theta(t) = \left(\sigma i t + \int_{(0, \infty)}(1 - e^{-i t x}) \nu(dx)\right).
    $$
    Ta predpis predstavlja bijekcijo med omenjenimi pari  $(\sigma, \nu)$ in neskončno deljivimi porazdelitvami na $\mathbb{R}^+$.
  \end{izrek}
\end{frame}

\begin{frame}
  \frametitle{Poissonova porazdelitev}
  Določimo par $(\sigma, \nu)$ za Poissonovo porazdelitev.
  \pause
  Vemo de je karakteristična funkcija Poissonove porazdelitve enaka
  $$
  \varphi(t) = \exp\left[\lambda(e^{it} - 1)\right].
  $$
  \pause
  Po Lévy-Hinčinovi formuli pa je oblike 
  \begin{align*}
  \varphi(t) &= \exp\left[-\sigma it - \int_{(0, \infty)}(1 - e^{-itx})\nu(dx)\right]= \\
             &= \exp\left[-\sigma it + \int_{(0, \infty)}(e^{-itx} - 1)\nu(dx)\right].
  \end{align*}
  \pause
  Vidimo lahko, da je $\sigma = 0$ in $\nu = \lambda \delta_1$.
\end{frame}

\end{document}