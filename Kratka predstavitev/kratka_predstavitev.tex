\documentclass[]{beamer} %zaradi [handout] \pause ne dela
\usepackage[T1]{fontenc}
\usepackage[utf8]{inputenc}
\usepackage[slovene]{babel}
\usepackage{pgfpages}
\usepackage{amsmath}
\usepackage{amssymb}
\usepackage{colortbl}
\usepackage{tikz}
\usepackage{array}
\usepackage{amsmath,amsthm, amsfonts,amssymb}
\usepackage{mathtools}
\usepackage{dsfont}

\setbeameroption{hide notes}
%\setbeameroption{show notes on second screen=right}

\mode<presentation>
\usetheme{Berlin}
\useinnertheme[shadows]{rounded}
\useoutertheme{infolines}
\usecolortheme{seahorse}
\usepackage{palatino}
\usefonttheme{serif}

%okolja za izreke, definicije, ...
\theoremstyle{plain}
\newtheorem{izrek}{Izrek}
\newtheorem{definicija}{Definicija}
\newtheorem{trditev}{Trditev}
\newtheorem{posledica}{Posledica}
\newtheorem{opomba}{Opomba}
\newtheorem{zgled}{Zgled}
\newtheorem{lema}{Lema}

%\beamertemplatenavigationsymbolsempty
\setbeamertemplate{headline}{}
%\setbeamertemplate{footline}{}

\title{Lévijevi procesi in njihova uporaba v financah}
\subtitle{}
\author[Anej Rozman]{Anej Rozman}
\institute[]{Mentor: doc.~dr. Martin Raič}
\date[]{}

\begin{document}


\frame{\titlepage}

\begin{frame}
  \frametitle{Vsebina}
  \begin{itemize}
    \item <1-> Neskončno deljive porazdelitve
    \item <2-> Lévijev proces
    \item <3-> (Uporaba v financah, Mertonov difuzijski model s skoki ?)
  \end{itemize}
\end{frame}


\begin{frame}
  \frametitle{Neskončno deljive porazdelitve}
  \begin{definicija}
    Pravimo, da ima realno številska slučajna spremenljivka $X$ \textit{neskončno deljivo porazdelitev}, če
    za vsak $n \in \mathbb{N}$ obstaja zaporedje neodvisnih enako porazdeljenih 
    slučajnih spremenljivk $\left(X_i\right)_{i=1,\dots,n}$, da velja
    $$
    X \stackrel{d}{=} X_1 + X_2 + \dots X_n,
    $$
    kjer $\stackrel{d}{=}$ pomeni enakost v porazdelitvi.
\end{definicija}


  \end{frame}


\begin{frame}
  \frametitle{Nekaj zgledov}
  \begin{zgled}
    Normalna porazdelitev je neskončno deljiva.
  \end{zgled}
  \begin{center}
    Naj bo $X \sim N(\mu, \sigma^2)$. Vemo da so linearne kombinacije neodvisnih normalno 
    porazdeljenih slučajnih spremenljivk spet normalno porazdeljene. Torej lahko za poljuben 
    $n\in\mathbb{N}$ zapišemo
    $$
      X \stackrel{d}{=} N(\frac{\mu}{n}, \frac{\sigma^2}{n}) + \dots + N(\frac{\mu}{n}, \frac{\sigma^2}{n}).
    $$
  \end{center}
\end{frame}


\begin{frame}
  \frametitle{Nekaj zgledov}
  \begin{zgled}
    Poissonova porazdelitev je neskončno deljiva.
  \end{zgled}
  \begin{center}
    Naj bo $X \sim \text{Pois}(\lambda)$. Vemo da so linearne kombinacije neodvisnih Poissonovo
    porazdeljenih slučajnih spremenljivk spet Poissonovo porazdeljene. Torej lahko za poljuben 
    $n\in\mathbb{N}$ zapišemo
    $$
      X \stackrel{d}{=} \text{Pois}(\frac{\lambda}{n}) + \dots + \text{Pois}(\frac{\lambda}{n}).
    $$
  \end{center}
\end{frame}

\begin{frame}
  \begin{lema}
    Linearne kombinacije neodvisnih neskončno deljivih slučajnih spremenljivk so neskončno deljive slučajne spremenljivke.
\end{lema}


\begin{proof}
    Za vsak $n \in \mathbb{N}$ lahko zapišemo $X_i \stackrel{d}{=} X_{i_1} + \dots + X_{i_n}$, torej za
    $a_1, \dots, a_n \in \mathbb{R}$ lahko $a_1X_1 + \dots + a_nX_n$ zapišemo kot 
    \begin{align*}
        a_1X_1 + \dots + a_nX_n &\stackrel{d}{=} \\
                                &\stackrel{d}{=} a_1(X_{1_1} + \dots + X_{1_n}) + \dots + a_n(X_{n_1} + \dots + X_{n_n}) \\
                                &\stackrel{d}{=} \sum_{i=1}^na_iX_{i_1} + \dots + \sum_{i=1}^na_iX_{i_n}.
    \end{align*}
\end{proof}
\end{frame}


\begin{frame}
  \frametitle{Lévy-Hinčinova formula}
  \begin{izrek}
    \textbf{(Lévy-Hinčinova formula)} Na $\mathbb{R}^+$ imamo bijekcijo med pari oblike
    $(\sigma, \nu)$ in množico momentno rodovnih funkcij neskončno deljivih porazdelitev,
    kjer je $\sigma \in \mathbb{R}^+$ in $\nu$ mera, ki zadošča pogoju $\int_{\mathbb{R}^+}1 \wedge x^2 \nu(dx)\leq \infty.$
    Momentno rodovna funkcija, ki ustreza paru $(\sigma, \nu)$ je oblike $M_{\sigma, \nu}(t) = e^{-\theta(t)}$,
    $$
      \theta(t) = \left(\sigma t + \int_{\mathbb{R}^+}1 - e^{-tx} \nu(dx)\right)
    $$
    
  \end{izrek}
\end{frame}

\begin{frame}
  \frametitle{Lévijev proces}
  \begin{definicija}
      Slučajnemu procesu $X = \{X_t \mid t \geq 0\}$ definiranem na verjetnostnemu
      prostoru $(\Omega, \mathcal{F}, \mathds{P})$ pravimo \textit{Lévijev proces}, če zadošča naslednjim pogojem:
      \begin{enumerate}
          \item $\mathds{P}(X_0 = 0)=1$.
          \item Trajektorije $X$ so $\mathds{P}$-skoraj gotovo zvezne z desne (z levimi limitami).
          \item Za $0 \leq s \leq t$ je $X_t - X_s$ enako porazdeljena kot $X_{t-s}$.
          \item Za $0 \leq s \leq t$ je $X_t - X_s$ neodvisna od $\{X_u \mid 0 \leq u \leq s\}$.
      \end{enumerate}
  \end{definicija}
\end{frame}

\end{document}